\chapter{Implementation}

This chapter extends the design presented in Chapter \ref{chap:design}, focusing on implementing the Proof Key for Code Exchange (PKCE) authorization flow for a dual-tenant environment. Using the multi-database architecture proposed in the design (refer to Figure \ref{fig:deployment_diagram_dual}), this implementation builds a foundational prototype of the OAuth 2.0 Authorization Code flow with PKCE, adapted for a multi-tenant, serverless infrastructure. By leveraging modern, cloud-native technologies—including Node.js, TypeScript, and LocalStack—this prototype demonstrates a secure, efficient, and scalable user authorization approach in a serverless multi-tenant application. The implementation focuses on the two critical endpoints: the authorization endpoint \textbf{/authorize} and the token endpoint \textbf{/token} deployed in a cloud infrastructure, which forms the cornerstone of the OAuth 2.0 PKCE flow. The following sections discuss the implementation details and the tests conducted for the project.


\section{Technical Stack}
The technical stack forms the foundations of any software development project, comprising a carefully selected combination of programming languages, frameworks, libraries, and tools that enable the creation of robust applications. During the prototyping phase, selecting an appropriate technical stack is particularly crucial, as it significantly influences development velocity, code quality, and the project's overall success. Moreover, security considerations are vital in stack selection, as well-established technologies often incorporate proven security practices and help mitigate potential vulnerabilities. To address these requirements and ensure a solid foundation for the prototype, the implementation leverages the following technology stack:

\begin{itemize}
    \item \textbf{Runtime Environment:} Node.js 18.x
    \item \textbf{Programming Language:} Javascript with TypeScript 5.6
    \item \textbf{Cloud Infrastructure:} AWS (simulated locally using LocalStack)
    \item \textbf{Database:} Amazon DynamoDB
    \item \textbf{Framework:} Serverless Framework
    \item \textbf{Infrastructure as Code:} Terraform and Terraform Local for localstack

\end{itemize}

\subsection{Development Dependencies}
A carefully selected set of dependencies was chosen to develop the prototype, enhancing the project's security, maintainability, testing, operability, logging, and performance. Each dependency was selected for a specific purpose, ensuring the system remains robust, efficient, and scalable throughout the development lifecycle. Table \ref{table:dev_dependencies} lists the dependencies used with its description. 

\begin{longtable}{|l|l|p{8cm}|}
    \caption{Project Dependencies}
    \label{table:dev_dependencies}
\hline
\rowcolor{grey!15}
\textbf{Dependency} & \textbf{Version} & \textbf{Description} \\ \hline
\endfirsthead
\hline
\rowcolor{grey!15}
\textbf{Dependency} & \textbf{Version} & \textbf{Description} \\ \hline
\endhead
\endfoot
\hline
\endlastfoot
argon2 & \textasciicircum 0.41.1 & A password hashing library implementing the Argon2 algorithm, designed for secure password storage. \\ \hline
axios & \textasciicircum 1.7.7 & A promise-based HTTP client for making HTTP requests in both Node.js and browsers. \\ \hline
esbuild & \textasciicircum 0.24.0 & A fast JavaScript and TypeScript bundler and minifier optimized for development and production builds. \\ \hline
fast-check & \textasciicircum 3.23.1 & A property-based testing library for generating random test cases and checking properties of your code. \\ \hline
hash-wasm & \textasciicircum 4.11.0 & A fast cryptographic hashing library using WebAssembly for efficiency. \\ \hline
jest-fuzz & \textasciicircum 0.1.2 & A fuzz testing library for Jest, used to test for unexpected edge cases. \\ \hline
jsonwebtoken & \textasciicircum 9.0.2 & A library to sign, verify, and decode JSON Web Tokens (JWT), commonly used for authentication and authorization. \\ \hline
uuidv4 & \textasciicircum 6.2.13 & A library to generate UUID (Universally Unique Identifier) version 4, primarily used for creating unique identifiers. \\ \hline
winston & \textasciicircum 3.15.0 & A flexible and popular logging library for Node.js, supporting multiple log transports. \\ \hline
winston-cloudwatch & \textasciicircum 6.3.0 & A transport for Winston that allows logging to AWS CloudWatch for centralized logging and monitoring. \\ \hline
zod & \textasciicircum 3.23.8 & A TypeScript-first schema validation library for parsing and validating inputs. \\ \hline
@aws-sdk/client-dynamodb & \textasciicircum 3.675.0 & AWS SDK Client for DynamoDB, used for interacting with Amazon DynamoDB from Node.js applications. \\ \hline
@eslint/js & \textasciicircum 9.14.0 & ESLint's core JavaScript configuration, used to enforce coding standards and best practices. \\ \hline
@types/aws-lambda & \textasciicircum 8.10.145 & TypeScript type definitions for AWS Lambda, providing type support for AWS Lambda functions. \\ \hline
@types/jest & \textasciicircum 29.5.14 & TypeScript type definitions for Jest, enabling type safety in Jest test cases. \\ \hline
@types/jsonwebtoken & \textasciicircum 9.0.7 & TypeScript type definitions for `jsonwebtoken`, helping with type safety when using JWTs in TypeScript code. \\ \hline
@types/node & \textasciicircum 22.7.7 & TypeScript type definitions for Node.js, providing type support for Node.js APIs in TypeScript projects. \\ \hline
@typescript-eslint/eslint-plugin & \textasciicircum 8.13.0 & ESLint plugin that provides TypeScript-specific linting rules. \\ \hline
aws-sdk-client-mock & \textasciicircum 4.1.0 & A mocking library for the AWS SDK, used in unit tests to mock AWS services like DynamoDB. \\ \hline
eslint & \textasciicircum 9.14.0 & A powerful linter for JavaScript and TypeScript that helps identify and fix coding issues. \\ \hline
eslint-plugin-security-node & \textasciicircum 1.1.4 & An ESLint plugin that identifies potential security vulnerabilities in Node.js code. \\ \hline
jest & \textasciicircum 29.7.0 & A JavaScript testing framework that supports unit tests, integration tests, and snapshot testing. \\ \hline
ts-jest & \textasciicircum 29.2.5 & A Jest transformer that allows Jest to run TypeScript code directly. \\ \hline
typescript & \textasciicircum 5.6.3 & The TypeScript programming language, which adds static types to JavaScript to improve developer productivity and code quality. \\ \hline
typescript-eslint & \textasciicircum 8.13.0 & A set of tools that allow ESLint to analyze TypeScript code and enforce best practices. \\ \hline
\end{longtable}

\subsection{Infrastructure as Code}
Infrastructure as Code (IaC) is an approach to managing IT infrastructures, like infrastructure or services created by AWS, using configuration and machine-readable definition files \citep{iac}. This method replaces the manual process of creating infrastructure and allows automation to provision different components like databases, servers, and networking components. One of the main advantages of IaC is that it provides consistency in provisioning infrastructure as a code-like script, leading to more reliable deployments. In addition, IaC also helps in faster deployments, and as the infrastructure is written in code, static analysis tools to check vulnerabilities are possible.

IaC is widely used in the industry to practice DevOps and continuously deploy. Many tools, such as AWS CloudFormation, AWS Cloud Development Kit (CDK), Ansible, and Terraform, are available to manage complex infrastructure. For this project, Terraform was chosen due to its status as one of the most widely used IaC tools, backed by a robust community and reusable modules \citep{terraform}.

Additionally, an essential aspect of this project is integrating the IaC tool with LocalStack, a platform that simulates AWS services locally for testing purposes. Terraform's compatibility with LocalStack is well-documented through tflocal, which allows developers to run Terraform configurations against LocalStack environments. This integration enables efficient local development and testing of infrastructure changes without needing access to actual AWS resources, further enhancing productivity and reducing costs during the development cycle. Refer to Listing \ref{lst:terraform_snippet} for an example where terraform is used in the project to create an API gateway resource.

\begin{lstlisting}[caption={Terraform Code Snippet for API Gateway}, label={lst:terraform_snippet}]
resource "aws_api_gateway_rest_api" "oidc_api" {
  name        = "oidc-apigw"
  description = "API Gateway for invoking Lambda"
  body = templatefile("${path.root}/../configurations/openapi.yml.tpl", {
    lambda_invoke_arn_authorize = module.pkce_authorize_lambda.aws_lambda_function.arn
    lambda_invoke_arn_token     = module.pkce_token_lambda.aws_lambda_function.arn
  })
  endpoint_configuration {
    types = ["REGIONAL"]
  }
}
\end{lstlisting}