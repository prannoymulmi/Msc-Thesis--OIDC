\newcommand{\abstractsection}{
    \vspace{-17em}
    \begin{center}
    \huge\textbf{Abstract}
    \end{center}
    \vspace{1em}
}

\begin{abstract}
\abstractsection
This study investigates the impact of urban green spaces on mental health outcomes in metropolitan areas. Using a mixed-methods approach, we analyzed data from 2,500 residents across 10 major cities. Our findings indicate a significant positive correlation between access to green spaces and improved mental well-being, with a 15\% reduction in reported stress levels for individuals living within 500 meters of a park or garden. Additionally, qualitative interviews revealed that regular interaction with nature in urban settings fostered a greater sense of community and social cohesion. These results underscore the importance of integrating green infrastructure in urban planning to promote public health and social resilience in increasingly densified city environments.
\end{abstract}