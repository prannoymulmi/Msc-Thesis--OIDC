\newcommand{\abstractsection}{
    \vspace{-10em}
    \begin{center}
    \huge\textbf{Abstract}
    \end{center}
}

\begin{abstract}
\abstractsection
OpenID Connect (OIDC), built on OAuth 2.0, is one of the popular protocol for authentication and authorisation in cloud environments. While cloud-based OIDC implementations offer scalability benefits, they face security challenges, including token security and Identity Provider protection. The integration of OIDC with cloud environments presents unique security risks that are evaluated using STRIDE threat modelling and risk matrix analysis. The risk analysis conducts a qualitative analysis on the threats on OIDC and cloud arranging them them in risk levels, taking the CAPEC database as a reference to assess the impact of the risk.

The assessment revealed significant vulnerabilities regarding cloud misconfiguration risks, information leakage potential, and GDPR compliance challenges. A prototype implementing security measures was developed using LocalStack to simulate the AWS environment, demonstrating practical mitigation strategies for these identified threats and testing the effectiveness of the mitigations. To validate the effectiveness of the mitigations, manual tests, automatic tests and Person's $\chi^2$  tests were conducted to test out information leakage during hash computation.
\newline\newline
\noindent \textbf{Keywords}: OpenID Connect, cloud computing, OIDC risks.
\end{abstract}

