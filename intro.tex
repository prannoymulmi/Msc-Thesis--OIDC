\chapter{Introduction}
\section{Background}
Cloud computing has undergone a significant transformation, especially for businesses, allowing them to create scalable, on-demand resources and easily accessible distributed systems for their applications. Services like accessing financial services, accessing health services, running powerplants, managing traffic, and managing public services are a few examples of where cloud technologies are in use \citep{intro_cloud_critical_infra}. This transformation is fundamentally changing the traditional analogue methods of self hosting infrastucture and replacing them with continual innovations, enhancing efficiency and improving user experience by providing a digital platform in the form of different applications/software hosted in the cloud. For example, in 2023, the German administration created a cloud strategy plan with the German Administration Cloud Strategy (DVS) to incorporate, strengthen and improve the status quo of digitalisation in the public sector \citep{german_gov_cloud_plan}. Such moves are not isolated to a single country and signal a move towards cloud and digitalisation to shape the new world. \newline

As this modernization reshapes the business landscape, there is a need to ensure that
only legitimate users gain access to sensitive data. Systems need to protect personal
data, and unauthorised breaches, to prevent loss of intellectual property,
violation of laws and regulations and disruption to public services \citep{critical_infra_reason}. Therefore, authentication and authorisation are fundamental concepts in cybersecurity, crucial for safeguarding digital resources and ensuring proper access control. \newline

Authentication is the process of verifying the identity of a user or an application using credentials like passwords, biometrics, or tokens \citep{authetication_intro}. On the other hand, authorisation determines the permissions or defines the granular access levels a user can execute, ensuring only the actions one is entitled to are used \citep{Gollmann2021-at}. Having both mechanisms to secure systems and avoid leakage of sensitive information and unauthorised access is especially important now due to the boom of the cloud-computing industry. Many services are easily accessible to a larger group of people, as running and creating different applications is easier, which simultaneously gives attackers more systems to compromise.\newline

In addition to the benefits that cloud computing provides, such as easy accessibility and quicker setup of applications, it has also introduced new challenges. In particular, the shared responsibility model of the cloud, where the cloud provider and the client both have specific responsibilities to secure the system \citep{shared_principal}. The shared responsibility model means that any misconfiguration or oversight on either side could leave the system vulnerable to attack. For instance, if the cloud provider maintains a secure infrastructure but the client misconfigures their application or leaves private databases open to the public, this could lead to data breaches or exploitation. This complication increases exploitable attack surfaces as different errors and misconfiguration could make the system more vulnerable. Therefore, integrating authentication and authorisation mechanisms for cloud-based applications has become increasingly vital. These processes ensure that sensitive information is protected and access is restricted solely to authorised entities who require it. \newpage


\section{Problem Statement}
Although cloud computing offers easy scalability, on-demand distributed system access and efficiency to its consumers, it presents significant security challenges, such as authentication and authorisation. Vulnerabilities in this area can lead to unauthorised access and data breaches, which are costly. Statista reported in 2023 that the average cost of a data breach is 4.45 million USD  \citep{statista_data_breach}. Given its widespread adoption, this thesis investigates the security risks of authentication and authorisation in cloud-based applications using the OpenID Connect (OIDC) protocol, a leading framework for Identity Management (IdM) supporting use cases such as mobile applications, machine-to-machine communication, and Single Sign-On (SSO)  \citep{oidc_popular}.


\section{Research Question}\label{sec:objectives}
This section presents the research question that this master's thesis is focused on. This main research question is:
\textbf{\textit{RQ1: What are the primary security concerns of the OpenID Connect protocol when used in a cloud-based application?}.}


\section{Aims and Objectives}
The primary aim of the RQ1 is to systematically analyse the security risks associated with implementing and using the OIDC protocol in cloud-based applications by identifying common vulnerabilities, assessing their impact on the overall security of these applications, and proposing effective mitigation strategies. Pursuing this aim, the thesis has the following objectives and goals:

\begin{enumerate}
  \item Identify Common Security Risks - 
  \begin{itemize}
      \item Evaluate at least five security risks for the cloud and five for OIDC protocol and describe their effects.
      \item Evaluate security risks only from research dating from 2020 to 2024.
  \end{itemize}

    
  \item Conduct Risk Assessment  - 
  \begin{itemize}
      \item Implement STRIDE threat modelling for the evaluated risks.
      \item Propose specific mitigation strategies for each identified threat.
      \item Create a risk matrix with probability and impact scores.
      \item Assess non-compliance risks for privacy law like the General Data Protection Regulation. 
  \end{itemize}
  \item Prototype Development and Validation - 
  \begin{itemize}
      \item Develop a working proof-of-concept incorporating identified security controls in a simulated cloud environment.
      \item Implement at least five of the identified mitigation strategies in the prototype.
      \item Achieve a minimum of 90\% code coverage in unit testing.
      \item Perform Manual and automated testing for the security controls implemented.
  \end{itemize}
\end{enumerate}

\section{Research Methodology}
The research will adhere to the Design Science Research Methodology (DSRM) for Information Systems Research. The DSRM is a robust framework for studying, creating, and evaluating IT artefacts. It is a suitable choice for this master thesis, which aims to design and develop artefacts for the OIDC protocol in a cloud-based application. This methodology is selected because it provides specific guidelines tailored for creating and implementing systems essential for conducting Design Science (DS) research in information systems \citep{dsrm}.

The DSRM framework encompasses six steps: problem identification, definition of objectives, design and development, evaluation, demonstration, and communication \citep{dsrm}. Figure \ref{fig:dsrm} depicts the six steps mentioned and how they would be used in this project and describes the implementation of this framework. These steps align seamlessly with the goals of this research. The primary aim of this project is to identify and evaluate potential risks associated with the OIDC protocol in a cloud-based application. Based on the insights gained from this evaluation, the project will proceed to design, create, and rigorously test a prototype. This approach ensures a structured and methodical examination of the OIDC protocol’s vulnerabilities and mitigations, thus aligning with the objectives outlined in section \ref{sec:objectives}.

In detail, the process begins with problem identification, where the specific security challenges and vulnerabilities of the OIDC protocol in cloud environments are identified. Following this, the objectives are defined, and clear goals for the research are set, particularly regarding risk assessment and artefact development. The design and development phase involves creating security artefacts, such as threat models based on the discovered risks.

The evaluation phase will assess these artefacts through various code and security testing methods to ensure their effectiveness and reliability. Demonstration involves showcasing the developed prototypes in practical scenarios to validate their functionality. Finally, the communication phase focuses on disseminating the findings and methodologies through comprehensive documentation and presentations, ensuring that the research contributes valuable insights to the broader field of information systems security.

By adhering to this structured methodology, the research addresses the immediate goals of evaluating and mitigating some risks associated with the OIDC. It provides a qualitative analysis that can inform future developments in the field. DSRM ensures that each research phase is meticulously planned and executed, ultimately leading to the development of robust security solutions for this use case.

\begin{figure}[h!]
\centering
\label{fig:dsrm}
\includegraphics[width=\textwidth, height=350px]{pics/dsrm.png}
\caption{DSRM Methodology}
\end{figure}
\newpage
\section{Thesis Outline}

This master's thesis project outlines six chapters that are as follows:

\begin{itemize}
    \item \textbf{Chapter 1} - The introduction chapter presents the background and the problem statement of this project and defines the research problems, research methodology, aim and objectives of this dissertation.\newline
    \textbf{Key Activities}
    \begin{itemize}
        \item Present research context and background on Cloud applications and OpenID Connect.
        \item Define specific problem statement regarding security challenges.
        \item Outline research methodology and approach.
        \item Define expected outcomes
    \end{itemize}

    \item \textbf{Chapter 2} - This chapter analyses existing work from research papers, journals, documentation, books, conference papers, theses and other scholarly sources concerning cloud application and OIDC. In addition to analysing and summarising the theories, a discussion identifies gaps in the existing literature for the applications using OIDC in the cloud environment.\newline

    \textbf{Key Activities}
    \begin{itemize}
        \item Analyse existing research on cloud applications.
        \item Review OIDC specifications and implementations.
        \item Examine security frameworks and protocols.
        \item Identify research gaps.
    \end{itemize}
    
    \item \textbf{Chapter 3} - Chapter 3 collects data from different case studies, academic journals, industry reports, and white papers on OIDC and the cloud. This data creates a threat model, including the mitigations for the modelled risks. \newline
    \textbf{Key Activities}
    \begin{itemize}
        \item Collect Common attack vectors for OIDC and cloud.
        \item Create comprehensive threat model and perform a risk assessment analysis.
        \item Propose mitigations for the identified risks.
    \end{itemize}
    
    \item \textbf{Chapter 4} - This is the design chapter, in which different architecture diagrams are created to implement the prototype. In addition to architecture diagrams, the application's setup methodology is depicted.\newline
   \textbf{Key Activities}
   \begin{itemize}
        \item Choose appropriate cloud provider.
        \item Define system architecture.
        \item Design security controls.
        \item Document setup methodology for the prototype.
    \end{itemize}
    
    \item \textbf{Chapter 5} - Chapter 5 continues the design based on Chapter 4's design, incorporating the security controls identified. This chapter also includes testing and evaluating the effectiveness of the implemented mitigations and overall functionality.\newline
       \textbf{Key Activities}
   \begin{itemize}
        \item Develop prototype application.
        \item Implement security controls.
        \item Conduct testing, such as unit testing, manual testing and automated security testing.
    \end{itemize}
    
    \item \textbf{Chapter 6} - Chapter 6 is the final section of this report, where the conclusions from the investigations and the future work that can be done to this research are presented.\newline
       \textbf{Key Activities}
        \begin{itemize}
            \item Summarize research findings.
            \item Propose future research directions.
        \end{itemize}

\end{itemize}