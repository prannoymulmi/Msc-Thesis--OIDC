\chapter{Introduction}
In the era of digitalisation, an immense transformation is commencing where day-to-day services such as shopping, communicating with people, visiting university courses, and also critical services like accessing financial services, accessing health services, running powerplants, managing traffic, and managing public services are moving digital \citep{intro_cloud_critical_infra}. This transformation is fundamentally changing the traditional analogue methods and replacing them with continual innovations, enhancing efficiency and improving user experience by providing a digital platform in the form of different applications/software hosted in the Cloud. As an example, recently, in 2023, the German administration created a cloud strategy plan with the German Administration Cloud Strategy (DVS) to incorporate and strengthen and improve the status quo of digitalisation in the public sector \citep{german_gov_cloud_plan}. Such moves are not isolated to a single country and signals a move towards cloud and digitalisation to shape the new world.

As this modernization reshapes the business landscape, there is a need to ensure that only legitimate users gain access to sensitive data and systems to safeguard personal data as well as unauthorized breaches, which can lead to loss of intellectual property, violation of laws and regulations and disruption to public services \citep{critical_infra_reason}. Therefore, authentication and authorization are fundamental concepts in cybersecurity, crucial for safeguarding digital resources and ensuring proper access control.

\section{Problem Definition}

\section{Research Question}

\section{Research Purpose}