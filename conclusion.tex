\chapter{Conclusion}

This thesis has provided an analysis of the security risks associated with the implementation and use of the OpenID Connect (OIDC) protocol in cloud-based applications on user-based tokens. By using a structured methodology, including threat modeling, risk assessment, and prototype development, the research has highlighted key vulnerabilities and proposed mitigation strategies to enhance the security of OIDC in cloud environments.

The findings emphasise that OIDC is a robust and widely adopted protocol for authentication and authorization, its security is highly dependent on proper implementation and configuration. Several critical risks were identified, including replay attacks, signature manipulation, malicious endpoint exploitation, and PKCE downgrade attacks. Moreover, the shared responsibility model inherent to cloud computing introduces additional challenges such as misconfigurations in Identity and Access Management (IAM) policies and multi-tenancy risks.

Security testing further demonstrated that cloud misconfigurations remain a significant concern. Whereas automated security testing tools such as Prowler and Snyk are valuable for identifying vulnerabilities, they are insufficient without expert review and adherence to best practices. Even with a well established protocols and cloud provides, OIDC and AWS respectively meticulous knowledge and right choice of tools are key to reduce the vulnerabilities.   


\section{Future Work}

\begin{itemize} 
    
    \item While this study mainly focused on user-based tokens and their associated security risks within the OpenID Connect (OIDC) protocol, it is important to acknowledge that OIDC also supports machine-to-machine (M2M) communication, also known as client authentication \citep{openid_docs}. In M2M scenarios, client authentication is crucial, as it involves server-to-server interactions without human intervention. Further research into the security risks of M2M communication, especially with cloud integration, would be beneficial to complement the findings of this study.

    \item Enhancing cyber hygiene and reducing cloud misconfigurations are is critical for future exploration. Addressing cloud misconfigurations resulting from human error or oversight, can significantly reduce the risk of unauthorised access and data breaches. A study which could provide effective measures against misconfigurations can greatly benefit in securing sensitive data.

    \item The prototype contains some limitations as the cloud environment was simulated using localstack. There are some real world scenarios which could not be tested. For example testing the domain name resolution to set specific headers for identifying tenants was not possible. Also local machine cannot cope with more load like the cloud making it not tested for a real-life scenario and its scalability. By leveraging AWS's infrastructure, one can test features like auto-scaling, and multi-region deployments under actual conditions, offering more accurate insights into performance and scalability. Although deploying on AWS incurs costs, it enables comprehensive testing with real-world data, ensuring that the application can handle production-level demands effective.
\end{itemize}






By expanding research into these areas, organisations can better secure their cloud environments and leverage OIDC for both user-based and machine-to-machine communications more effectively.


